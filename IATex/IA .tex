\documentclass{beamer}

\usepackage[francais]{babel}
\usepackage[utf8]{inputenc}
\usepackage[T1]{fontenc}

\usetheme{Warsaw}
\title{Intelligence Artificielle : promesses et réalités}
\author{Ouail Abed - Hackenolz Guillaume - Soufyani Amine}
\date{20 mars 2018}


\begin{document}

	\begin{frame}
	\titlepage
	\end{frame}
	
	\begin{frame} % 2eme transparent TDM générale
	\frametitle{Table des matieres}
	\tableofcontents[hideallsubsections] %ou [pausesections]
	\end{frame}

	
	\begin{frame}[fragile]
	\frametitle{Intelligence Artificielle : Definition}
	\begin{itemize}
		\item Intelligence Artificielle (IA)
		  est « l'ensemble de théories et de techniques mises en œuvre en vue de réaliser des machines     					  capables de simuler l'intelligence ». Elle correspond donc à un ensemble 	de concepts et de 						  technologies plus qu'à une discipline autonome constituée.
	\end{itemize}
	\end{frame}
	
	\begin{frame}[fragile]
	\frametitle{Intelligence Artificielle : Definition}
	\begin{itemize}
		\itemsep2em
		\item Souvent classée dans le groupe des sciences cognitives, elle fait appel à la neurobiologie 					computationnelle (particulièrement aux réseaux neuronaux), à la logique mathématique et à 							l'informatique. 
		
		\item Un réseau neuronal est l’association, en un graphe plus ou moins 															complexe,d’objets élémentaires, les neurones formels qui sont eux mêmes inspirées du fonctionement 		des neuronnes biologiques
	\end{itemize}
	\end{frame}
	
	
	\begin{frame}[fragile]
	\frametitle{2 Types d'IA}
	\begin{itemize}
		\itemsep0.5em
		\item IA faible  :\\
		est une intelligence artificielle non-sensible qui se concentre sur une tâche précise
		 
		 \item IA forte: \\
		est une intelligence artificielle dotée de conscience, de sensibilité et d'esprit
		\item les systèmes actuellement existants sont considérés comme des intelligences artificielles faibles
	\end{itemize}
	\end{frame}
	
	\begin{frame}[fragile]
	\frametitle{Breve Histoire de l'IA}
	\begin{itemize}
		\itemsep0.5em
		\item (1943) La naissance des ordinateurs :\\
		 Les premiers ordinateurs voient le jour. Construits avec des technologies qui précédaient les circuits intégrés (tubes à vide, relais électromécaniques), ils sont peu performants.
		 
		 \item (1950) Le test Turing : \\
		 Le mathématicien britannique Alan Turing publie son article "Computing Machinery and Intelligence" et met au point son test à l’aveugle pour déterminer qui est l’humain ou l’ordinateur.

		 \item (1950) La première machine capable d’apprendre :\\
		 Claude Shannon développe Theseus, une souris électromécanique capable d’apprendre à trouver la sortie d’un labyrinthe. Avant même l’apparition du terme "intelligence artificielle", il s’agissait de la première démonstration effective d’une machine capable d’apprendre.

	\end{itemize}
	\end{frame}
	
	\begin{frame}[fragile]
	\frametitle{Breve Histoire de l'IA(suite)}
	\begin{itemize}
		\itemsep1em
		\item (1956) Le séminaire du Dartmouth College :\\
		 Les premiers ordinateurs voient le jour. Construits avec des technologies qui précédaient les circuits intégrés (tubes à vide, relais électromécaniques), ils sont peu performants.
		 
		 \item (1958) Le « list processing » : \\
		 John McCarthy, co-organisateur du séminaire du Dartmouth College, créé le langage informatique LISP (mot forgé à partir de l’anglais "list processing") qui permet de faciliter la programmation d’IA.

		 \item (1959) Le « General Problem Solver » :\\
		 Herbert Simon et Allen Newell inventent le General Problem Solver, une stratégie de résolution de problèmes largement utilisée dans le domaine de l'intelligence artificielle.

	\end{itemize}
	\end{frame}
	
	\begin{frame}[fragile]
	\frametitle{Breve Histoire de l'IA(suite)}
	\begin{itemize}
		\itemsep1em
		\item (1965) Le programme Eliza :\\
		 Eliza est un programme informatique écrit par Joseph Weizenbaum, capable de dialoguer en anglais en incarnant le rôle d’une psychologue.

		 \item ((1974) Le système MYCIN : \\
		 MYCIN est un système expert utilisant l’IA pour identifier des bactéries causant des infections 	sévères et recommander des antibiotiques en adaptant le dosage au poids des patients.

		 \item (1996) La victoire Deep Blue :\\
		 Le champion d’échecs Garry Kasparov est battu par le superordinateur Deep Blue d’IBM. Un événement qui démontre que l’IA est plus performante que l’homme dans certains domaines précis.

	\end{itemize}
	\end{frame}
	
	\begin{frame}[fragile]
	\frametitle{Breve Histoire de l'IA(suite)}
	\begin{itemize}
		\itemsep1em
		\item (2005) Le robot Stanley :\\
		 En 2005, Stanley, un robot construit à l’université Stanford, remporte le "DARPA Grand Challenge" en conduisant de manière autonome pendant 131 miles sur une piste de désert sans avoir fait de reconnaissance préalable
		 
		 \item (2001) Le programme Watson :\\
		 Le programme d’IA Watson d’IBM surclasse les meilleurs joueurs du jeu télévisé américain de questions réponses Jeopardy !

		 \item (2017) L’AlphaGo :\\
		 En mars 2016, le programme d’IA de Google AlphaGo bat un des meilleurs joueurs mondiaux de jeu de go, puis le 27 mai 2017, il bat le champion du monde Ke Jie.

	\end{itemize}
	\end{frame}
	
	\begin{frame}[fragile]
	\frametitle{Évolution de L'intelligence  Artificielle}
	
	\centerline{\includegraphics{evolution.png}}
	
	\end{frame}
	
	\begin{frame}[fragile]
	\frametitle{Évolution de L'intelligence  Artificielle : Machine Learning}
	\begin{itemize}
		\item Le machine learning permet à une machine d’adapter ses comportements en se fondant sur l’analyse des données à sa disposition. Un robot peut ainsi apprendre à marcher en commençant par des mouvements aléatoires, puis en sélectionnant les mouvements lui permettant d’avancer.
	\end{itemize}
	\end{frame}
	
	\begin{frame}[fragile]
	\frametitle{Évolution de L'intelligence  Artificielle : Deep Learning}
	\begin{itemize}
		\itemsep1em
		\item Le deep learning est la branche du machine learning qui utilise comme modèles mathématiques les réseaux de neurones formels, eux-mêmes construits sur la représentation mathématique et informatique d’un neurone biologique, née en 1943.
		\item  Le Deep Learning est utilisé dans la voiture autonome  de Google : le réseau de  neurones 	classifie tout l’environnement pour éviter les obstacles ou s’arrêter au bon moment 
	\end{itemize}
	\end{frame}
	
	\begin{frame}[fragile]
	\frametitle{Évolution de L'intelligence  Artificielle}
	
	\centerline{\includegraphics{deeplearning.png}}%changer pour image de meilleure qualitée
	
	\end{frame}
	
		\begin{frame}[fragile]
	\frametitle{Évolution de L'intelligence  Artificielle (fin)}
	\begin{itemize}
		\item Ces deux branches de l'intelligence artificielle ont causées de grandes améliorations de algorithmes, mais malgres cela l'IA d'aujourd'hui est est toujours qualifiée de « faible », en opposition à l’IA « forte » et consciente d’elle-même que prédisent les transhumanistes.

	\end{itemize}
	\end{frame}
	
	\begin{frame}[fragile]
	\frametitle{Les Dangers de l'Intelligence Artificielle}
	\begin{itemize}
	\itemsep1em
		\item Il existe plusieurs inquiétudes sur l'IA
		\item En 2014, Stephen Hawking met en garde sur le risque que l'IA devienne plus intelligente que l'Homme et le domine
		\item Moshe Vardi, un spécialiste américain de l'informatique, suppose que l'IA pourrait mettre 50\% de l'humanité au chômage
		\end{itemize}
	\end{frame}
	
	\begin{frame}[fragile]
	\frametitle{Les Dangers de l'Intelligence Artificielle}
	\begin{itemize}
	\itemsep1em
		\item En Février 2018, 26 experts spécialistes en intelligence artificielle mettent en garde contre les dangers d'un usage criminelle de l'IA : augmentation de la la cybercriminalité , conduire à des utilisations de drones ou de robots à des fins terroristes, etc...
		\item Selon eux, dans les dix prochaines années, l'efficacité croissante de l'IA risque de renforcer la cybercriminalité mais aussi de conduire à des utilisations de drones ou de robots à des fins terroristes
		\item Moshe Vardi, un spécialiste américain de l'informatique, suppose que l'IA pourrait mettre 50\% de l'humanité au chômage
		\end{itemize}
	\end{frame}

	\begin{frame}[fragile]
	\frametitle{Les Dangers de l'Intelligence Artificielle}
	GOOGLE HOME
	\centerline{\includegraphics[height=8cm]{googlehome.png}}
	\end{frame}
	
	\begin{frame}[fragile]
	\frametitle{Les Dangers de l'Intelligence Artificielle}
	DRONE
	\centerline{\includegraphics[height=6cm]{drone.png}}
	\end{frame}
\end{document}